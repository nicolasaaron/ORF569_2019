\subsection{cut norms}

\begin{definition}
	The cut norm of a graphon $W \in \tilde{\mathcal{W}}$ is denoted by $\Vert W \Vert_{\Box}$ and is defined as follows:
	\begin{equation}
	\Vert W \Vert_{\Box} = \sup_{D_1, D_2 \in [0,1]} \left\vert \int_{D_1 \times D_2} W(x,y) dx dy\right\vert
	\end{equation}
	where $D_1, D_2$ are measurable subsets of $[0,1]$. The cut metric between two graphons $V,W \in \tilde{\mathcal{W}}$ is given by
	$$
	d_\Box(W,V) = \inf_{\pi \in \prod_{[0,1]} } \Vert W^{\pi} - V \Vert_{\Box},
	$$
	where $W^{\pi}(x,y) = W(\pi(x), \pi(y))$ and $\prod_{[0,1]}$ is the class of measure preserving permutations $\pi: [0,1] \to [0,1]$.
\end{definition}

\begin{remark}
	By Monotonicity argument, we can have
	$$
	\Vert W \Vert_{\Box} = \sup_{0 \leq f \leq 1, 0 \leq g \leq 1} \left\vert \int \int f(x)g(y) w(x,y) dx dy \right\vert.
	$$
\end{remark}

Because of the permutation function $\phi$, the mapping $d_\Box$ is not a well defined metric in $\tilde{\mathcal{W}}$. We define the space $\mathcal{W} = \tilde{\mathcal{W}} / \sim$
where the equivalent relation $\sim$ on $\tilde{W}$ is defined by
$$
V \sim W \text{ if there exists } \pi \in \prod\,_{[0,1]} \text{   such that } V = W^{\pi}.
$$
Namely, we identify graphons up to measure preserving permutations. THus, the space $(\mathcal{W}, d_\Box)$ is a well defined metriczable vector space.

\begin{proposition}
	The space $(\mathcal{W}, d_\Box)$ is complete.
\end{proposition}

\begin{lemma} (\cite{janson2010graphons} Theorem E.6)
	Given a graphon $W \in \mathcal{W}$ and the associating graphon operator $A^W$, if $|W| = \int W(x,y) dy \leq 1$, then we have for any $p,q \in [1,\infty]$ and $q' = (1- 1/q)^{-1}$,
	$$
	\Vert W \Vert_{\Box} \leq \Vert A^W \Vert_{L^p, L^q} \leq \sqrt{2} (4 \Vert W \Vert_{\Box})^{(1-\frac{1}{p}) \wedge \frac{1}{q}}
	$$
	where
	$$ \Vert A^W \Vert_{L^p, L^q} =  \sup_{ \Vert f\Vert_{L^p} \leq 1} \Vert A^W f \Vert_{L^q} = \sup_{ \Vert f \Vert_{L^p} \leq 1}\, \sup_{ \Vert g \Vert_{L^{q'}}  \leq 1} \left\vert \int A^W f(x) \, g(x) dx \right\vert.
	$$	
	A special case is when $p=q=2$, then we have
	$$
	\Vert W \Vert_{\Box} \leq \Vert A^W \Vert_{L^2, L^2} \leq \sqrt{8 \Vert W \Vert_{\Box}}.
	$$	
	\label{lemma:cut_norm_ineq}
\end{lemma}


The proof of Lemma \ref{lemma:cut_norm_ineq} relies on the Riesz-Thorin interpolation theorem and the inequality $$	\Vert W \Vert_{\Box} \leq \Vert A^W \Vert_{L^{\infty}, L^1} \leq 4 \Vert W \Vert_{\Box}.$$
We first recall the Riesz-Thorin interpolation theorem: 
\begin{theorem}(Riesz-Thorin interpolation theorem)\\
	Suppose $(p_0, q_0), (p_1, q_1)  \in [1,\infty] \times [1, \infty]$ and for some $r \in (0,1)$ such that
	$$ \left( \frac{1}{p}, \frac{1}{q} \right) = r  \left(\frac{1}{p_0}, \frac{1}{q_0} \right) + (1-r) \left( \frac{1}{p_1}, \frac{1}{q_1} \right).$$
	Consider a linear function $T$ that maps $L^{p_0}( (X, \mu), \mathbb{C})+ L^{p_1}( (X, \mu), \mathbb{C})$ into $L^{q_0}((Y, \nu), \mathbb{C}) + L^{q_1}( (Y, \nu), \mathbb{C})$, where $(X,\mu)$ and $(Y,\nu)$ are two measurable spaces. If $q_0 = q_1 = \infty$, we further suppose that the measure $\nu$ is semifinite.
	If we assume that there exists two constants $M_0, M_1 >0$ such that
	$$
	\Vert T f \vert_{L^{q_0}} \leq M_0 \Vert f \Vert_{L^{p_0}}, \qquad \forall \, f\in L^{p_0}( (X,\mu), \mathbb{C})
	$$
	and 
	$$
	\Vert T f \vert_{L^{q_1}} \leq M_1 \Vert f \Vert_{L^{p_1}}, \qquad \forall \, f\in L^{p_1}( (X,\mu), \mathbb{C}),
	$$
	then $T$ is bounded on $L^p$ and furthermore, 
	$$
	\Vert T f \Vert_{L^q} \leq M_0^{1-r} M_1^{r} \Vert f \Vert_{L^{p}}
	$$
	for any complex-value function $f \in L^{p}(X,\mu, \mathbb{C})$.
\end{theorem}

\begin{lemma}
	$$
	\Vert W \Vert_{\Box} \leq \Vert A^W \Vert_{L^{\infty}, L^1} \leq 4 \Vert W \Vert_{\Box}.
	$$
\end{lemma}
\begin{proof}
	\begin{eqnarray}
	\Vert A^W \Vert_{L^\infty, L^1} &=& \sup_{ \Vert f \Vert_{L^\infty} \leq 1}\, \sup_{ \Vert g \Vert_{L^{\infty}}  \leq 1}  \left\vert \int A^W f(x) g(x) dx  \right\vert \nonumber \\
	&=& \sup_{ -1 \leq f \leq 1} \sup_{-1 \leq g \leq 1 }  \left\vert \int \int W(x,y)f(y) g(x) dxdy \right\vert \label{eq:cut_ineq_p_infty_q_1_lower}
	\end{eqnarray}
	Take $f= \mathbbm{1}_{D_1}$ and $g=\mathbbm{1}_{D_2}$ for $D_1, D_2 \subset [0,1]$, then we have
	$$
	\Vert A^W \Vert_{L^\infty, L^1} \geq \Vert W \Vert_{\Box}.
	$$
	
	We can introduce two functions $f_1,f_2$ valued in $[0,1]$ such that $f= f_1- f_2$, as well as $g_1,g_2$ such that $g=g_1 - g_2$. Thus,
	\begin{equation}
	\Vert A^W \Vert_{L^\infty, L^1} = \sup_{ 0 \leq f_1, f_2 \leq 1 }\, \sup_{ 0 \leq g_1, g_2 \leq 1} \left\vert \int \int W(x,y)[f_1(y)- f_2(y)]\cdot [g_1(x) - g_2(x)] dxdy \right\vert
	\label{eq:cut_ineq_p_infty_q_1_upper}
	\end{equation}
	Since
	\begin{eqnarray*}
		&& \left\vert \int \int W(x,y)[f_1(y)- f_2(y)]\cdot [g_1(x) - g_2(x)] dxdy \right\vert \\
		&=& \left\vert \langle A^W (f_1 - f_2), (g_1-g_2) \rangle \right\vert \\
		&=& \left\vert \langle A^W f_1, g_1 \rangle + \langle A^W f_2, g_2 \rangle - \langle A^W f_2, g_1 \rangle - \langle A^W f_1, g_2 \rangle \right\vert \\
		&\leq& | \langle A^W f_1, g_1 \rangle | + | \langle A^W f_2, g_2 \rangle | + | \langle A^W f_2, g_1 \rangle | + | \langle A^W f_1, g_2 \rangle| 
	\end{eqnarray*}
	We have
	$$
	\Vert A^W \Vert_{L^\infty, L^1} \leq \sup_{ 0 \leq f_1, f_2 \leq 1 }\, \sup_{ 0 \leq g_1, g_2 \leq 1} | \langle A^W f_1, g_1 \rangle | + | \langle A^W f_2, g_2 \rangle | + | \langle A^W f_2, g_1 \rangle | + | \langle A^W f_1, g_2 \rangle|  = 4 \Vert W \Vert_{\Box}.
	$$
\end{proof}


\begin{proof}(for Lemma \ref{lemma:cut_norm_ineq})
We show in the first place that $\Vert W \Vert_{\Box} \leq \Vert A^W \Vert_{L^{p}, L^{q} }$. Since $L^{\infty}([0,1], dx) \subset L^p([0,1], dx)$ and $L^{q}([0,1], dx) \subset L^1([0,1],dx)$, we have
\begin{eqnarray*}
	\Vert A^W \Vert_{L^{p}, L^{q}} &=& \sup_{ \Vert f \Vert_{L^{p}} \leq 1 } \Vert A^W f \Vert_{L^{q} }\\
	&\geq & \sup_{ \Vert f \Vert_{L^\infty} \leq 1 } \Vert A^W f \Vert_{L^{q}} \\
	& \geq & \sup_{ \Vert f \Vert_{L^\infty} \leq 1 } \Vert A^W f \Vert_{L^1} \\
	&=& \Vert A^W \Vert_{L^{\infty}, L^1}\\
	&\geq& \Vert W \Vert_{\Box}
\end{eqnarray*}
	
For the upper bound, we let $r = (1-\frac{1}{p})\wedge \frac{1}{q}$, then $1-r = \frac{1}{p} \vee (1-\frac{1}{q})$. We define $p_1 =  (1-r)p$ and $q_1 = \left(1 - (1-r)^{-1} ( 1- 1/ q) \right)^{-1}$. then $\overline{p}, \overline{q} \in [1,\infty]$. Thus, with $p_0 = \infty$ and $q_0 = 1$, we have
$$
	\left( \frac{1}{p}, \frac{1}{q} \right) = r  \left(\frac{1}{p_0}, \frac{1}{q_0} \right) + (1-r) \left( \frac{1}{p_1}, \frac{1}{q_1} \right).
$$
Then by applying Riesz-Thorin interpolation theorem, we obtain
$$
	\Vert A^W \Vert_{L^{p}, L^{q} } \leq 	\Vert A^W \Vert_{L^{p}(\mathbb{C}), L^{q}(\mathbb{C}) } \leq \left( \Vert A^W \Vert_{L^{\infty}(\mathbb{C}), L^1(\mathbb{C})} \right)^r \cdot \left( \Vert A^W \Vert_{L^{p_1}(\mathbb{C}), L^{q_1}(\mathbb{C}) } \right)^{1-r}.
$$
where $L^{\infty}(\mathbb{C}), L^{1}(\mathbb{C}), L^{p_1}(\mathbb{C}), L^{q_1}(\mathbb{C})$ are complex-value function spaces extended from the corresponding function spaces.

We can easily see that
\begin{eqnarray*}
	\Vert A^W \Vert_{L^\infty(\mathbb{C}), L^1(\mathbb{C})} 
	&=& \sup_{\stackrel{ f: [0,1] \to \mathbb{C}}{\Vert f \Vert_{L^\infty} \leq 1} }  \sup_{\stackrel{ g: [0,1] \to \mathbb{C}}{\Vert g \Vert_{L^\infty} \leq 1} } \left\vert \int \int W(x,y) f(y) g(x) dy dx \right\vert \\
	&\leq&  2 \Vert A^W \Vert_{L^{\infty}, L^1} 
\end{eqnarray*}
Indeed, this inequality can be improved to a constant $\sqrt{2}$ (see \cite{krivine1979constantes})
$$
	\Vert A^W \Vert_{L^\infty(\mathbb{C}), L^1(\mathbb{C})} \leq  \sqrt{2} \Vert A^W \Vert_{L^{\infty}, L^1}
$$

For any $p_1,q_1 \in [1,\infty]$,
\begin{eqnarray*}
	\Vert A^W \Vert_{L^{p_1}(\mathbb{C}), L^{q_1}(\mathbb{C})} &=& \sup_{ \Vert f \Vert_{L^{p_1}(\mathbb{C})} \leq 1 } \Vert A^W f \Vert_{L^{q_1}(\mathbb{C}) }\\
	&\leq & \sup_{ \Vert f \Vert_{L^1(\mathbb{C})} \leq 1 } \Vert A^W f \Vert_{L^{q_1}(\mathbb{C})} \\
	& \leq & \sup_{ \Vert f \Vert_{L^1(\mathbb{C})} \leq 1 } \Vert A^W f \Vert_{L^{\infty}(\mathbb{C})} \\
	&=& \Vert A^W \Vert_{L^{1}(\mathbb{C}), L^\infty(\mathbb{C})}
\end{eqnarray*}
where the first inequality is justified by $L^1([0,1], dx, \mathbb{C}) \subset L^{p_1}([0,1], dx, \mathbb{C}) $ and the second inequality comes from $\Vert f \Vert_{L^{q_1}(\mathbb{C})} \leq \Vert f \Vert_{L^\infty(\mathbb{C})}$.
Moreover,
\begin{eqnarray*}
	\Vert A^W \Vert_{L^1(\mathbb{C}), L^\infty(\mathbb{C})} &=& \sup_{ f: [0,1] \to \mathbb{C},\, \Vert f \Vert_{L^1} \leq 1 } \Vert A^W f \Vert_{L^\infty(\mathbb{C})}\\
	&= &  \sup_{ \stackrel{f: [0,1] \to \mathbb{C}}{ \Vert f \Vert_{L^1} \leq 1} }  \sup_{ \stackrel{g: [0,1] \to \mathbb{C}}{\Vert g \Vert_{L^1} \leq 1} }  \left\vert \int \int W(x,y)f(y) g(x) dy dx \right\vert \\
	&\leq& \sup_{ f: [0,1] \to \mathbb{C}, \,  \Vert f \Vert_{L^1} \leq 1 }  \left\vert \int \sup_{ g: [0,1] \to \mathbb{C}, \,  \Vert g \Vert_{L^1} \leq 1 } \left\vert \int W(x,y)g(x)dx \right\vert \cdot  f(y) dy \right\vert \\
	&\leq& \sup_{ f: [0,1] \to \mathbb{C}, \,  \Vert f \Vert_{L^1} \leq 1 } \Vert W \Vert_{L^\infty(\mathbb{C})} \cdot  \int \vert f(y)  \vert dy \\
	&=&  \Vert W \Vert_{L^\infty(\mathbb{C})} \\
	&=& ess\sup_{x \in [0,1]} \left\vert \int W(x,y) dy \right\vert \\
	&\leq & 1
\end{eqnarray*}
where the second inequality comes from Fubini's theorem and the last inequality comes from the assumption on graphon kernel $|W| \leq 1$.

Hence, with the fact that $\sqrt{2}^r \leq \sqrt{2}$ and $\Vert A^W \Vert_{L^{\infty}, L^1} \leq 4 \Vert W \Vert_{\Box}$, we conclude that
$$
	\Vert A^W \Vert_{L^{p}, L^{q}} \leq \sqrt{2} \left( 4 \Vert W \Vert_{\Box} \right)^{\min \{ 1- \frac{1}{p}, \, \frac{1}{q} \} }.	
$$	
\end{proof}

\subsection{Centrality measure}
An extension of Bonacich measure. Far a graphon $W \in \cW$ associated with an operator $A^W$ and a value $\alpha \in \mathbb{R}$. Assume that $\alpha \in [0, \frac{1}{\rho(A^W)} )$, and we define a mapping $b_\alpha$ by
\begin{equation}
\begin{array}{rccl}
	b_\alpha: &[0,1] &\longrightarrow & \mathbb{R}\\
	& x & \mapsto & b_\alpha(x) := \left( \left[ \mathbb{I} - \alpha A^W ]^{-1} 1_{[0,1]} \right] \right)(x)
\end{array}
\end{equation}
The function $b_\alpha$ is called Bonacich centrality function.


\subsection{some words on the mean field interaction}

Graphon games are with an infinity of players. We can look at N player games and consider the case when $N \to \infty$. We assume that players are symmetric. Let us denote by $A$ the set of admissible controls for a single player where $A$ is some subset in $\mathbb{R}$.  For any strategy profile $\balpha \in (A)^N$, recall that the cost function associated to player $i$ is defined by
\begin{equation}
	J^i(\balpha) = J^i(\alpha^i, (\alpha^1, \ldots, \alpha^{i-1}, \alpha^{i+1}, \ldots, \alpha^N)) = J^i(\alpha, \balpha^{-i})
\end{equation}

An important special case is when $J^i$ is symmetric with respect to all the $\alpha^j$, $j\neq i$. Indeed, in this case, we can rewrite the cost function for player $i$ into a new cost function $\tilde{J}^i$ which involves a mean filed interaction term. More precisely, we have
\begin{equation}
	J^i(\alpha, \balpha^{-i}) = \tilde{J}^i\left(\alpha^i, \frac{1}{N-1} \sum_{j\neq i} \delta_{\alpha^j} \right) = \tilde{J}^i\left(\alpha^i, \mu^{\balpha^{-i}} \right)
\end{equation}
where $\mu^{\balpha^{-i}} := \frac{1}{N-1} \sum_{j\neq i} \delta_{\alpha^j} \in \mathcal{P}_2(A)$ is a squared integrable measure on $A$ and $\tilde{J}^i : A \times \mathcal{P}_2(A) \ni (\alpha, \mu) \mapsto J^i(\alpha, \mu) \in \mathbb{R}$ represents the new cost function associated to player $i$.\\

\textbf{Example: ``Where do we put the towel on the beach?"  }

There are $N$ players, $A$ is a compact set of $\mathbb{R}$ and for player $i$, her control $\alpha^i \in A$ represents her position on the beach. Let $d: A \times A \to \mathbb{R}$ be a distance function. For player $i$, her cost function $J^i$ takes the form
$$
	J^i(\balpha) = a \cdot d(\alpha^i, \alpha_0) - b \cdot \frac{1}{N-1} \sum_{j \neq i} d(\alpha^i, \alpha^j)
	= a \cdot d(\alpha^i, \alpha_0) - b \cdot \int_{A} d(\alpha^i, \alpha) \mu^{\balpha^{-i}}(d\alpha)
$$
where $a, b > 0$ and $\alpha_0 \in A$ is the preferred position for all players.

The objective is to solve for $ONE$ typical player when she plays against the distribution of all other players.

The best response function is defined as
$$
	\hat{\alpha} : \mathcal{P}_2(A) \ni \mu \mapsto \arginf_{\alpha \in A} J(\alpha, \mu)  \subseteq A 
$$
and a Nash equilibrium is defined as a distribution $\hat{\mu} \in \mathcal{P}_2(A)$ that satisfies the fixed point argument:
$$
	\supp(\hat{\mu}) \subseteq \hat{\alpha}(\hat{\mu}),
$$
where $\supp(\mu)$ stands for the support of a distribution $\mu\in \mathcal{P}_2(A)$. 



\subsection{Graphon game}

In finite player games, players can be enumerated by a finite set of numbers, say $\{1, \ldots, N\}$ for example, and, a strategy profile can be by a vector $\balpha =(\alpha_1, \ldots, \alpha_N) \in A^1 \times \ldots A^N$ with $A^i \subseteq \mathbb{R}^k$. Now, when dealing with Graphon games, we consider a continuum of players such that a single player is represented by a real number $x\in [0,1]$, whilst a strategy profile turns out to be a squared integrable mapping $\balpha : [0,1] \ni x \mapsto \balpha(x) \in R^k$. The set of feasible strategies for player $x\in[0,1]$, denote by $A(x)$ is a subset of $\mathbb{R}^{k}$, and the collection of all feasible strategy sets among players is denote by $A = \{A(x)\}_{x\in [0,1]}$. We can also define the set of admissible strategy profiles as 
$$
\mathcal{A} = \left\{ \balpha \in L^{2}([0,1], \mathbb{R}^k) \, : \quad \forall x \in [0,1], \, \balpha(x) \in A(x) \right\}.
$$


In finite Network game, player $i$ feels the interaction from player $j$ through the Adjacency matrix $A^{(G)}$, and the local aggregated perceived by player $i$ can be modeled through a term 
$$
	z^i(\balpha) = \frac{1}{N-1} \sum_{j \neq i} A_{ij}^{(G)} \alpha^j.
$$
Similarly, in Graphon games, we define the local aggregate of interaction perceived by player $x \in [0,1]$ with respect to a strategy profile $\balpha = (\balpha_1,\ldots, \balpha_k)^\top \in L^2([0,1], \mathbb{R}^k)$ by
$$
	z(x | \balpha) = \int_{0}^{1} W(x,y) \balpha(y) dy \in \mathbb{R}^k
$$
where $z(x|\balpha)_j = \int_{[0,1]} W(x,y) \balpha_j(y) dy =( A^W \balpha_j )(x)$ for every $j=1, \ldots, k$ and every $x\in [0,1]$.

We define a new graphon operator $\mathbb{A}^W: L^2([0,1], \mathbb{R}^k) \to L^2([0,1], \mathbb{R}^k)$ such that for any strategy profile $\balpha \in L^2([0,1], \mathbb{R}^k)$ and for any $x \in [0,1]$, we have
$$
	\left( \mathbb{A}^W \balpha \right)(x) := z(x | \balpha) = \left[ \begin{array}{c} A^W \balpha_1 (x) \\ \vdots \\ A^W \balpha_k (x)  \end{array} \right].
$$

For player $x \in [0,1]$, her cost function is defined by 
$$
J^x: A(x) \times \mathcal{A} \ni (\alpha' , \balpha) \mapsto J(\alpha', z(x|\balpha))
$$
where $J: \mathbb{R}^k \times \mathbb{R}^k \ni (\beta, z) \mapsto J(\beta,z) \in \mathbb{R}$ is a predefine cost function for the graphon game. 

\begin{definition}
	A graphon game $\mathcal{G}(A, J, W)$ is defined in terms of a continuum set of players indexed by $[0,1]$, a graphon $W \in \mathcal{W}$, a cost functions $J(\cdot, \cdot)$, and a collection of feasible strategy sets $A = \{A(x)\}_{x \in [0,1]}$. 
\end{definition}

The best response function for player $x\in [0,1]$ in a graphon game is define as a set value function
$$
	\underline{\alpha}_{BR}^x : \mathcal{A} \ni \balpha \mapsto \arginf_{\beta \in A(x)} J(\beta, z(x|\balpha))
$$
An admissible strategy $\balpha^* \in \mathcal{A}$ is a Nash equilibrium if it satisfies a fixed point condition:
$$
\balpha^*(x) \in \underline{\alpha}^x_{BR}(\balpha^*) \quad \textbf{for all } x \in [0,1], 
$$
in another words, for every $x \in [0,1]$, 
$$
	J(\balpha(x), z(x | \balpha^*)) \leq J( \alpha', z(x|\balpha^*) ), \qquad \alpha' \in A(x).
$$



\begin{remark}
If we assume that for any $z \in \mathbb{R}^k$, there exists a unique minimum for the mapping $\beta \mapsto J(\beta, z)$, then we can see that the best response function for graphon game $\mathcal{G}(A,J,W)$ as a function from $\mathcal{A}$ to $\mathcal{A}$ such that
$$
	\underline{\alpha}_{BR} : \mathcal{A} \ni \balpha \mapsto \left(x \mapsto \arginf_{\beta \in A(x) } J\left(\beta, z(x | \balpha) \right) \right).
$$

Since $\balpha \in \mathcal{A} \subseteq L^2([0,1], \mathbb{R}^k)$, we need to show that $\underline{\alpha}_{BR}$ does not depend on the representation of $\balpha$ and it is also squared integrable on $[0,1]$. This first point can be shown from the fact that $\mathbb{A}^W\balpha$ is independent of the choice of representative $\balpha$ in $L^2([0,1], \mathbb{R}^d)$. The second point can be justified assuming the feasible strategy set $A(x)$ is a subset of a compact set $A_{comp}$ of $\mathbb{R}^k$. 
\end{remark}

\ \\
Example: Min-Max graphon: let $W(x,y) = x \wedge y (1- x \vee y)$ for any $(x,y) \in [0,1]\times [0,1]$.


\begin{remark}
	If $f \in ker(A^W - \lambda I)$, then $(f, 0,0,\ldots,0)$ is an eigenvector of $\mathbb{A}^W$ corresponded to the eigenvalue $\lambda$.
\end{remark}


\subsubsection{existence result}

\begin{assumption}\ 
	\begin{enumerate}
	%	\item $A(x) = \mathbb{R}^k$ for every $x \in [0,1]$.
		\item The function $J : \mathbb{R}^k \times \mathbb{R}^k \ni (\alpha,z)\to J(\alpha,z) \in \mathbb{R}$ is continuously differentiable and strongly convex in $\alpha$ with a uniform constant $l_c > 0$ for every $z \in \mathbb{R}^k$. Namely, for every $\alpha', \alpha \in \mathbb{R}^k$ and $z \in \mathbb{R}^k$, we have for every $b \in [0,1]$
		$$
			J(b \alpha + (1-b) \alpha' , z) \leq b J(\alpha,z) + (1-b) J(\alpha', z) - \frac{l_c}{2} b (1-b) \Vert \alpha' - \alpha \Vert^2 .
		$$
		\item $\nabla_\alpha J(\alpha, z)$ is Lipschitz in $z$ with a uniform constant $l_J$ for every $\alpha \in \mathbb{R}^k$. Namely, for every $z',z \in \mathbb{R}^d$ and for every $\alpha \in \mathbb{R}^k$, we have
		$$
			\Vert \nabla_\alpha J(\alpha, z') - \nabla_\alpha J(\alpha, z) \Vert \leq l_J \Vert z' - z \Vert.
		$$		
		\item For every $x\in [0,1]$, the set of feasible strategy $A(x)$ is closed and convex. 
	\end{enumerate}
\label{assumption:Garphon_games_1}
\end{assumption}


 
\begin{definition}
	We define an operator $\mathbb{B}: L^2([0,1], \mathbb{R}^k) \ni \tilde{z} \mapsto \mathbb{B}\tilde{z} \in L^2([0,1], \mathbb{R}^k)$ such that  for every $x \in [0,1]$
	$$	
		\mathbb{B}\tilde{z}(x) = \arginf_{\alpha \in A(x)} J(\alpha, \tilde{z}(x)).
	$$
\end{definition}

We can see that
$$
	\balpha \text{ is a Nash Equilibrium} \Longleftrightarrow \balpha \text{ is a fixed point of } \mathbb{B} \mathbb{A}^W.
$$

\begin{remark}
	$\mathbb{B}$ is not a linear operator.
\end{remark}

\begin{lemma}
Under assumption \ref{assumption:Garphon_games_1}, we can show that $\mathbb{B}$ is Lipschitz. More precisely, for every $f,g \in L^2([0,1], \mathbb{R}^k)$,
$$
	\Vert \mathbb{B} f - \mathbb{B} g \Vert_{L^2} \leq \frac{l_J}{l_c} \Vert f - g \Vert_{L^2}.
$$
\label{lemma:continuity_of_B}
\end{lemma}

\begin{proof}
	Under the strong convexity assumption, we know that for every $f \in L^2([0,1], \mathbb{R}^k)$ and every $x \in [0,1]$, the set $\mathbb{B}f(x)$ that minimizes $J(\cdot, f(x))$ is a singleton. By the abuse of terminology, we still denote it by $\mathbb{B}f(x)$.
	
	By variation inequality, the strong convexity of $J$ in $\alpha$ implies that for every $f,g \in L^2([0,1], \mathbb{R}^k)$ and $x \in [0,1]$, 
	\begin{eqnarray*}
		\langle \nabla_\alpha J(\mathbb{B}f(x), f(x)) , \, \mathbb{B}g(x)- \mathbb{B}f(x)  \rangle&\geq & 0 \\
		\langle \nabla_\alpha J(\mathbb{B}g(x), g(x)) , \, \mathbb{B}f(x)- \mathbb{B}g(x) \rangle &\geq & 0.
	\end{eqnarray*}
	We add up these two inequalities and obtain that
	\begin{eqnarray*}
		&&\langle \nabla_\alpha J(\mathbb{B}f(x), f(x)) - \nabla_\alpha J(\mathbb{B}f(x), g(x) ) , \,   \mathbb{B}g(x)- \mathbb{B}f(x) \rangle  \\
		&\geq& 	\langle \nabla_\alpha J(\mathbb{B}g(x),  g(x)) - \nabla_\alpha J(\mathbb{B}f(x), g(x) ) ,\,  \mathbb{B}g(x)- \mathbb{B}f(x) \rangle
	\end{eqnarray*}
	The strong convexity of $J$ in $\alpha$ also implies the strong monotonicity of $\nabla_\alpha J$ in $\alpha$, namely for any $\alpha, \alpha', z \in \mathbb{R}^k$,
	$$
		\langle \nabla_\alpha J(\alpha, z) - \nabla_\alpha J(\alpha',z),\, \alpha - \alpha' \rangle \geq l_c \Vert \alpha - \alpha' \Vert^2
	$$
	Thus, we have
	\begin{eqnarray*}
		&& \Vert \nabla_\alpha J(\mathbb{B}f(x), f(x)) - \nabla_\alpha J(\mathbb{B}f(x), g(x) )  \Vert \cdot \Vert  \mathbb{B}g(x)- \mathbb{B}f(x) \Vert  \\
		&\ge & 	\langle \nabla_\alpha J(\mathbb{B}f(x), f(x)) - \nabla_\alpha J(\mathbb{B}f(x), g(x) ) , \,   \mathbb{B}g(x)- \mathbb{B}f(x) \rangle \\
		& \geq & \langle \nabla_\alpha J(\mathbb{B}g(x),  g(x)) - \nabla_\alpha J(\mathbb{B}f(x), g(x) ) ,\,  \mathbb{B}g(x)- \mathbb{B}f(x) \rangle \\
		&\geq & l_c \Vert \mathbb{B}g(x) - \mathbb{B}f(x) \Vert^2
	\end{eqnarray*}
	where the first inequality is justified by the Cauchy-Schwartz inequality.
	
	Since $\nabla_\alpha J$ is Lipschitz in $z$, so we can have
	$$
		\Vert \mathbb{B}g(x) - \mathbb{B}f(x) \Vert \leq \frac{l_J}{l_c} \Vert f(x) - g(x) \Vert 
	$$
	Hence, 
	\begin{equation*}
		\Vert  \mathbb{B}g - \mathbb{B}f \Vert^2_{L^2} = \sum_{i=1}^k \int_{[0,1]} \left( [\mathbb{B}f(x)]_{i} - [\mathbb{B}g(x)]_{i} \right)^2 dx 
		\leq  \left( \frac{l_J}{l_c} \right)^2 \Vert f - g \Vert^2_{L^2}
	\end{equation*}
\end{proof}

{\color{red} The following assumption guarantees that for every admissible strategy profile $\balpha \in \mathcal{A}$, the best response function evaluated at $\balpha \in \mathcal{A}$, $\underline{\alpha}_{BR}(\balpha)$, is squared integrable thus belongs to $\mathcal{A}$. }
\begin{assumption}
	There exists a $M > 0$ such that for all $x \in [0,1]$, $ \sup_{\alpha \in A(x) } \Vert \alpha \Vert \leq M$.
\label{assumption:Graphon_games_2}
\end{assumption}


In order to proof the existence of a Nash equlibrium, we would like to use the Schauder's fixed point theorem: let $F: B \mapsto B$ and $L$ be a closed and convex subset of $B$, if $F(L) \subset L$, $F$ continuous, and $F(L)$ belongs to a compact subset of $B$. then there exists a fixed point $b \in L$ such that $b = F(b)$.


\begin{lemma}
 Suppose that the graphon game $\mathcal{G}(A,J,W)$ satisfies assumptions \ref{assumption:Garphon_games_1} and \ref{assumption:Graphon_games_2}, then it admits at least one Nash equilibrium.
\end{lemma}

\begin{proof}
	Let us define the set $L_A = \{ f \in L^2([0,1], \mathbb{R}^k), \Vert f \Vert \leq M \}$.
	
	\begin{itemize}
		\item We know that the operator $\mathbb{B}\mathbb{A}^W$ maps $L^2([0,1], \mathbb{R}^k)$ to $L^2([0,1], \mathbb{R}^k)$. And since $\mathbb{B} : L^2([0,1], \mathbb{R}^k) \to L^2([0,1], \mathbb{R}^k)$ and $\mathbb{A}^W : L^2([0,1], \mathbb{R}^k) \to L^2([0,1],\mathbb{R}^k)$ are continuous operators, so as $\mathbb{B} \mathbb{A}^W$. 
		\item $L_A$ is non-empty, convex, and compact (bounded and closed) set of $L^2([0,1], \mathbb{R}^k)$.
		\item $\mathbb{A}^W$ is a compact operator. Thus $\mathbb{A}^W L_A$ is pre-compact in $L^2([0,1], \mathbb{R}^k)$. Thus, $\mathbb{B} \overline{\mathbb{A}^W L_A}$ is compact. So $\mathbb{B}\mathbb{A}^W L_A$ belongs to a compact subset of $L^2([0,1], \mathbb{R}^d)$.
		\item Since for every $f \in L^2([0,1], \mathbb{R}^k)$ and every $x \in [0,1]$, $\mathbb{B}f(x) = \arginf_{\alpha \in A(x)} J(\alpha, f(x)) \in A(x)$, so that $\mathbb{B}f \in L_A$ under assumption \ref{assumption:Graphon_games_2}. Thus $\mathbb{B}\mathbb{A}^W L_A \subseteq L_A$.
	\end{itemize}
	
	Therefore, we can apply Schauder's fixed point theorem on $L_A$ with respect to the function $\mathbb{B}\mathbb{A}^W$.
\end{proof}



\subsubsection{uniqueness result}

We can relax assumption \ref{assumption:Graphon_games_2} by 
\begin{assumption}
	There exists $\hat{z} \in \mathbb{R}^k$ and $M >0$ such that for every $x \in [0,1]$ and for every $\hat{\alpha} \in \argmin_{\alpha' \in A(x)} J(\alpha', \hat{z})$,
	$$
		\Vert \hat{\alpha} \Vert \leq M.
	$$	
\label{assumption:Graphon_games_3}
\end{assumption}

\begin{lemma}
	Under assumption \ref{assumption:Garphon_games_1} and \ref{assumption:Graphon_games_3}, we have for every $f \in L^2([0,1], \mathbb{R}^k)$, $\mathbb{B}f \in L^2([0,1], \mathbb{R}^k)$.
\end{lemma}

\begin{proof}
	It is easy to see that for a constant function $g(x) = \hat{z}$ for every $x\in [0,1]$,
	$$
		\Vert \mathbb{B} g \Vert_{L^2}^2 = \int_{[0,1]} \Vert \argmin_{\alpha' \in A(x)} J(\alpha', \hat{z}) \Vert^2 dx \leq M^2
	$$
	Thus, for every $f \in L^2([0,1], \mathbb{R}^k)$, we have
	$$
		\Vert \mathbb{B}f \Vert_{L^2} \leq \Vert \mathbb{B}f - \mathbb{B} g \Vert_{L^2} + \Vert \mathbb{B}g \Vert_{L^2} \leq \frac{l_J}{l_c} \Vert f - g \Vert_{L^2} + M < \infty. 
	$$
\end{proof}
\begin{assumption}
	Suppose that
	$$
		\frac{l_J}{l_c} \lambda_{max}(W) \leq 1,
	$$
	where $\lambda_{max}$ is the largest eigenvalue of $W$.
\label{assumption:Graphon_games_4}
\end{assumption}

\begin{lemma}(Uniqueness)
	Suppose that the graphon game $\mathcal{G}(A,J, W)$ satisfies Assumption \ref{assumption:Garphon_games_1}, \ref{assumption:Graphon_games_3}, and  \ref{assumption:Graphon_games_4}, then it admits a unique Nash equilibrium.
\label{lemma:Graphon_game_unique_NE}
\end{lemma}

\begin{proof}
	We can show that the game operator $\mathbb{B} \mathbb{A}^W$ is a contraction, then the result follows from Banach fixed point theorem. 
\end{proof}
 