\section{Model for Games with a continuum of players}

\begin{definition}\ 
	\begin{itemize}
		\item Player: $(I, \mathcal{J}, \nu)$ is a probability space where $I$ represents the set of players, $\mathcal{J}$ is a $\sigma$-algebra of coalitions, and $\nu$ is a probability distribution that models the weights put on coalitions;
		\item $E$ is a Banach space (for example $\mathcal{C}_0([0,1])$, $\mathbb{R}^k$, etc);
		\item there exists a function $A: I \to 2^E$ such that for $\nu-$almost every $i \in I$, $A(i)$ is the set of feasible actions for player $i$.
		\item $(\Omega, \mathcal{F}, \mathbb{P})$ is a probability space where $\Omega$ is the set of state of nature, $\mathcal{F}$ represents the events, and $\mathbb{P}$ is a probability distribution.
	\end{itemize}
\end{definition}

We define the set of admissible controls by 
$$
	L_A = \{\alpha\in L^1( (I, \mathcal{J},\nu), E) | \, \alpha(i) \in A(i) \quad \nu-a.e. \, i\in I \}.
$$

We define the ``Bochner integral" on the Banach space $E$ of an admissible control $\underline{\alpha} \in L_A$ by
$$
	\mathbb{E}[\alpha] = \int_{\Omega} \alpha(i) d\nu(i)  \in E.
$$

\begin{definition}
We define a preference function $\Pi : I \times \Omega \times L_A \ni (i, \omega, \alpha) \mapsto \Pi(i,\omega, \alpha) \in 2^E$ such that
$$
	\Pi(i,\omega, \alpha) \subset A(i),\quad \text{for } \nu-a.e.\, i\in I \text{ and for } \mathbb{P}-a.e. \, \omega \in \Omega.
$$	
The set $\Pi(i,\omega, \alpha)$ represents the set of all actions that are feasible for player $i$ and preferred to the actions $\alpha(i) \in A(i)$, given that the state of the world is $\omega \in \Omega$ and the other players $j \neq i$ who play with actions $(\alpha(j))_{j\neq i}$.
\end{definition}

We assume that players observe the state of nature $\omega \in \Omega$, and they decide to take actions according to a decision rule:
$$
\underline{\alpha} : \Omega \ni \omega \mapsto \underline{\alpha}(\omega) \in L_A.
$$


\begin{definition}
	A decision rule $\underline{\alpha}^*$ is a (random) Cournot-Nash equilibrium (CNE for short) if for $\mathbb{P}$-a.e. $\omega \in \Omega$ and $\nu-$a.e. $i \in I$
	$$
		\Pi(i, \omega, \underline{\alpha}^*(\omega)) = \emptyset.	
	$$
\end{definition}

Example: If we have associate to player $i$ a cost function $J^i : \Omega \times A(i) \times L_A (\omega, \alpha_i, \alpha) \mapsto J^i(\omega, \alpha_i, \alpha) \to \mathbb{R}$. Then $\underline{\alpha}^*$ is a CNE if for $\mathbb{P}$-a.e. $\omega \in \Omega$ and $\nu-$a.e. $i\in I$, we have for all $\alpha_i \in A(i)$,
$$
	J^i(\omega, \underline{\alpha}^*(\omega)(i), \underline{\alpha}^*(\omega) ) \leq J^i( \omega, \alpha_i, \underline{\alpha}^*(\omega)).
$$
And in this case, for any decision rule $\underline{\alpha}: \Omega \to L_A$, we have for $\nu$-a.e. $i\in I$ and $\mathbb{P}$-a.e. $\omega \in \Omega$,
$$
	\Pi(i, \omega, \underline{\alpha}) = \left\{ \alpha_i \in A(i) |  \, J^i(\omega, \alpha_i, \underline{\alpha}) < J^i( \omega, \underline{\alpha}(\omega)(i), \underline{\alpha}(\omega) ) \right\}.
$$


\begin{remark}
	For a continuous measure $\nu$, if $\forall i \in I$, $\nu(\{i\}) =0$ , then $I$ cannot be countable.
	
	Definition of an atom: a set $A$ is said to an atom if $A \in \mathcal{J}$, $\nu(A) >0$, and for any $B \subseteq A$ and $B \in \mathcal{J}$, we have
	$$
		\nu(B) = 0 \quad or \quad \nu(B) = \nu(A).
	$$	
\end{remark}


\begin{remark}
	Case of finite games: $I=\{1,\ldots, N\}$. We have $\nu(A) = \frac{1}{N} | A |$ for any $A \subset I$, and 
	$$
		\int_I f(x) dx = \frac{1}{N} \sum_{i=1}^N f(i).
	$$
\end{remark}

\begin{remark}
For more references, see Aumann and Mas Colell. And also G. Carmona for non-atomic games and atomic games.
\end{remark}


%%%%%%%%%%%%%%%%%%%%%%%%%%%%%%%%%%%%

\subsection{Coordination and Acquisition: Beauty contest}
Example: ``Beauty contest" (Morris and Shin)\\

Background description: 

we assume that there is a continuum of players. The demands are
\begin{itemize}
	\item uncertainty (size of customer base);
	\item supplier price;
	\item average price of the other supplier.
\end{itemize}
The improvement of decision making relies on the survey that collections the market information (acquisition part). The information consists of different signals, and can be exogenous or indigenous.

The Game is a one-shot game, in which players take actions simultaneously. 

A player is represented by a real number $l \in [0,1]$. Player $l$ chooses $n$ sources of information acquisitions $z_l \in \mathbb{R}^n_+$. For each source $i\in \{1,\ldots,n\}$, $z_{li}$ represents the attention paid by player $i$ to the source $i$. 

After the choice $z_l$, player $i$ observes signals $x_l =(x_{l1}, \ldots, x_{ln}) \in \mathbb{R}^n$ about an unobservable quantity $\theta$. The attention $z_{li}$ is the precision of signal $x_{li}$.

Player $l$ then takes action $\alpha_l$.

The cost function to player $l \in [0, 1]$ is defined by
$$
u_l = C(z_l) + (1-\gamma) (\alpha_l - \theta)^2 _ \gamma (\alpha_l - \overline{\alpha})^2
$$
where $C: \mathbb{R}_+ \to \mathbb{R}$ is differentiable and increasing, $\gamma \in [-1, 1]$ and $\overline{\alpha}$ represents the average action, namely
$$
	\overline{\alpha} = \int_0^1 \alpha_l dl \in \mathbb{R}.
$$

\begin{remark}\ 
	\begin{itemize}
		\item The term $(\alpha_l - \overline{\alpha})^2$ represents the coordination part between players. If $\gamma >0$, each player wants to choose a similar action as her peers . Otherwise, if $\gamma < 0$, each player intends to behave differently than others.
		\item In order to make the average action $\overline{\alpha}$ well defined, we need to assume that the function $[0,1] \ni l \mapsto \alpha_l$ is measurable and integrable.
	\end{itemize}
\end{remark}

%%%%%%%%%%%%%%%%%%%%%%%%%%%%%%%%%%%%%%%%%%
\subsubsection{signals}
Assume that player don't know the value of $\theta$ (i.e. player starts with no knowledge of $\theta$). The signal observed by player $l$ to the source $i$ is modeled by 
$$
	x_{li} = \theta + \eta_i + \epsilon_{li}
$$
where $\eta_i \sim \mathcal{N}(0, \kappa_i^2)$ and $\epsilon_{li} \sim \mathcal{N}(0, \frac{\xi_{i}^2}{z_{li}})$, and $k_i, \xi_i$ are constant.

The term $\eta_i$ stands for the noise displayed by the source $i$ when player $l$ approaches to it. The term $\epsilon_{li}$ represents the noise brought by the action $\alpha_l$.

We also notice that the variance of $\epsilon_{li}$, say $\frac{\xi_{i}^2}{z_{li}}$, is inversely proportional to the precision $z_{li}$. In another words, the more attention player $l$ gives to signal $i$, the less the noise incurred by the action of player $l$.

\begin{assumption}
	Suppose that $\eta_i$ and $\epsilon_{li}$ for all $i \in \{1\ldots,N\}$ and $l \in [0,1]$ are all independent.
\end{assumption}

%%%%%%%%%%%%%%%%%%%%%%%%%%%%%%%%%%%%%%
\subsubsection{correlation}
Let us look at the covariance of observations of the same source from different players. Suppose $l \neq l'$, then
\begin{eqnarray}
	cov(x_{li}, x_{l'i}) &=& \mathbb{E}\left[ (x_{li} - \mathbb{E}[x_{li}]) (x_{l'i}- \mathbb{E}[x_{l'i}]) \right] \nonumber  \\
	&=& \mathbb{E}[(\eta_i + \epsilon_{li}) (\eta_i + \epsilon_{l'i}) ] \nonumber \\
	&=& \kappa_i^2 \nonumber \\
	&=& \rho_{ll' i} \sigma_{li} \sigma_{l'i}
\end{eqnarray} 
where 
$$
	\sigma_{li}^2 = \kappa_i^2 + \frac{\xi^2_i}{z_{li}}, \qquad \sigma_{l'i}^2= \kappa_i^2 + \frac{\xi_{i}^2}{z_{l'i}}, \qquad \text{and } \quad \rho_{ll'i} = \frac{k_i^2}{\sigma_{li} \sigma_{l'i} }.
$$

\begin{remark}\ 
	\begin{itemize}
		\item If a player pays more attention to information, then the precisions of the corresponding observed signals increase ($z_{li}$ increases), which will also lead to an increase of the correlation $\rho_{ll'i}$ between this player and the others.
		\item When $\xi_0$ or $\zeta_{li} \to \infty$, then $\rho_{ll' i}  \to 1$. In this case, we say that the signals are public, i.e. the observation $x_{li}$ does not depends on player $l$.
		\item When $\kappa_i = 0$, then $\rho_{lli} =0$, we say in this situation that the signals are private.
	\end{itemize}
\end{remark}


%%%%%%%%%%%%%%%%%%%%%%%%%%%%%%%%%%%%%%%%%%%%%%%%%
\subsubsection{look for symmetric equilibrium}

\begin{definition}
For player $l\in [0,1]$, a strategy is a couple $(z_l ,a_l)$ where $z_l \in \mathbb{R}^n_+$ and $a_l : \mathbb{R}^n \to \mathbb{R}$. A strategy $(z_l, a_l)$ is of feedback form if the action taken by player $l$ satisfies
$$
	\alpha_l = a_l(x_l). 
$$
where $x_l$ is the observation of the precision $z_l$.
\end{definition}

\begin{definition}
	The game is said to be symmetric if all player $l \in [0,1]$ use a common strategy $(z, a)$.
\end{definition}

For any given $l \in [0,1]$, suppose that player $l$ choses a strategy $(z_l,a_l)$ and for all other players $l' \neq l$, they follow a strategy $(z_{l'}, a_{l'}) = (z,a)$, then the cost function for player $l$ with strategy $(z_l, a_l)$ is given by
$$
	J^l( (z_l, a_l), (z_{-l}, a_{-l}) ) = \mathbb{E}[u_l] = C(z_l) + (1-\gamma) \mathbb{E}[(\alpha_l - \theta)^2] + \gamma \mathbb{E}[ (\alpha_l - \overline{\alpha})^2]
$$
where
$$
	\overline{\alpha} = \int_{0}^1 a_{l'}(x_{l'}) dl'.
$$

\begin{remark}
	The definition of $\overline{\alpha}$ seems to say that it should be a random variable because $x_l$ are random variables for all $l \in [0,1]$.
\end{remark}

\begin{remark}
	Think about the case in $N$ players game, with observations $x_1, \ldots, x_N$ and $l\in \{1,\ldots,N\}$. We can still assume that $x_{li} = \theta + \eta_i + \epsilon_{li}$. Then 
	$$
		\overline{\alpha}^N = \frac{1}{N} \sum_{l=1}^N a_l(x_l).
	$$
	Suppose that $(z_l, a_l) = (z,a)$ for all $l\in \{1,\ldots,N\}$, then by law of large numbers, when $N \to \infty$, 
	$$ \overline{\alpha}^N \longrightarrow \mathbb{E}[a(x)] \in \mathbb{R}$$
	where $x=(x_1, \ldots, x_n)$ with $x_i \sim \mathcal{N}(0, \kappa_i^2 + \frac{\xi_i^2}{z})$.
	
	However, we expect that $\overline{\alpha}$ to be random variable instead of a real number. Here, we need to clarify the Law of large number for continuum of random variables.
	
	For each $l\in [0,1]$, Let $X_l = a(x_l)$ be a random variable on $(\Omega, \mathcal{F}, \mathbb{P})$ valued in $\mathbb{R}$. Then, for every $\omega \in \Omega$,
	$$
		\overline{\alpha} = \int_0^1 X_l(\omega) dl.
	$$
	Thus, assume that we can apply the Fubini's theorem on $\Omega \times [0,1] \ni (\omega, l) \mapsto X_l(\omega) = a(x_l(\omega)) \in \mathbb{R}$, then
	\begin{align*}
		\mathbb{E}[\overline{\alpha}] &= \int_\Omega \int_0^1 X_l(\omega) dl d\mathbb{P}(\omega) \\
		&=\int_0^1 \int_{\Omega} X_l(\omega) dP(\omega) dl \\
		&= \int_0^1 \mathbb{E}[X_l] dl \\
		&= \mathbb{E}[X_l]
	\end{align*}
	
	This means that the Law of large number for continuum random variables seems to be equivalent to the Fubini's theorem. 
	
	However, the function $ [0,1] \times \Omega \ni (l, \omega) \mapsto X_l(\omega) = a(x_l(\omega)) \in \mathbb{R}$ is not necessarily jointly measurable on the product space $[0,1] \times \Omega$ with respect to the product $\sigma-$algebra $ \mathcal{J} \times \mathcal{F}$.
\end{remark}

\begin{remark} 
	\ 
	\begin{itemize}
	\item I want all the random variable $X_l$ for $l \in [0,1]$ to be independent.
	\item I want $X_l$ to have the same distribution $\mu$.
	\end{itemize}

	Let us denote $\Omega = (\mathbb{R}^n)^{I}$ with $I=[0,1]$, the set of all function from $I$ to $\mathbb{R}^n$. Consider a probability measure $\mu$ on $(\mathbb{R}^n, \mathcal{B}(\mathbb{R}^n))$ where $\mathcal{B}(\mathbb{R}^n)$ is the Borel $\sigma-$algebra on $\mathbb{R}^n$. Consider $p$ real numbers $J_p = (l_1, \ldots, l_p) \in I^p$, and a projection function $\pi_{J_p}: \Omega \ni x = (x_l)_{l\in I} \to (x_{l_1}, \ldots, x_{l_p}) = x_{J_p} \in (\mathbb{R}^n)^p$, 
	Let us denote by $\mu_{J_p}$ the distribution of $\pi_{J_p}(x)$ on $(\mathbb{R}^n)^p$. If $(x_j)_{j \in J_p}$ are independent following the distribution $\mu$, then $\mu_{J_p} = \mu \otimes \ldots \otimes \mu$. 
	
	For any finite subsets $J_{p_1}, J_{p_1}$ of $I$ such that $J_{p_1} \subset J_{p_2}$, the i.i.d assumption guarantees the consistency condition, namely 
	$$
		\mu_{J_{p_1}} = \mu_{J_{p_2}} \circ \pi_{J_{p_2} \to J_{p_1}}^{-1}
	$$  
	where $\pi_{J_{p_2} \to J_{p_1} }$ is the projection mapping from $J_{p_2}$ to $J_{p_1}$.
	
	The Kolmogorov extension theorem tells us that there must exists a unique probability measure on $\Omega$, denote by $\mathbb{P}_\mu$ such that
	$$
		\mathbb{P}_\mu( x_{J_p}  \in A) = \mu_{J_p}(A), \qquad \forall \, A \in (\mathcal{B}(\mathbb{R}^n) )^p.
	$$
	Hence, the probability space $(\Omega, (\mathcal{B}(\mathbb{R}^n) )^I, \mathbb{P}_\mu)$ is well defined.		
	However, we do not have measurability for the function 
	$$
		[0,1] \times \Omega \ni (l, \omega) \mapsto X_l(\omega) = \omega(l) \in \mathbb{R}^n.
	$$
	This is usually called the White Noise.
\end{remark}


\subsubsection{Linear strategy in ``Beauty contest"}

Assume that for all $l \in [0,1]$, the strategy $a_l : \mathbb{R}^n \to \mathbb{R}$ is linear in $x$ and coefficient sum up to $1$, i.e. $a_l(x) = \sum_{i=1}^n a_{li} x_i$ with $\sum_{i=1}^n a_{li} = 1$.
Then
\begin{equation}
	\overline{\alpha} =  \int_{0}^1 a_{l'}(x_{l'}) dl' = \int_0^1 \left( \sum_{i=1}^n a_{l'i} x_{l'i} \right) dl'.
\end{equation} 

We look for symmetric equilibrium, name for strategy $a: \mathbb{R}^n \ni x \mapsto \sum_{i=1}^n a_i x_{li} \in \mathbb{R}$. Since $x_{li} = \theta + \eta_i + \epsilon_{li}$, we deduce that
\begin{align*}
	\overline{\alpha} &= \int_0^1 \left( \sum_{i=1}^n a_i (\theta + \eta_i + \epsilon_{l'i} ) \right) dl' \\
	&= \theta + \sum_{i=1}^n a_i \eta_i + \int_0^1 \left(\sum_{i=1}^n a_i \epsilon_{l'i} \right) dl' \\
	&= \theta + \sum_{i=1}^n a_i \eta_i
\end{align*}
where the last equality is justified by the fact that 
$$
	\int_{0}^1 \epsilon_{l'i} dl' = 0.
$$

Hence, for a fixed $l \in [0,1]$, player $l$ with strategy $(z_l,a_l)$, and for all $l' \neq l$ such that $(z_{l'}, a_{l'}) = (z, a)$, the cost function of player $l$ becomes 
\begin{align*}
	J^l( (z_l, a_l), (z_{-l}, a_{-l}) ) &= C(z_l) + (1-\gamma) \mathbb{E}[ (a_l(x_l) - \theta)^2 ] + \gamma \mathbb{E}[ (\alpha_l(x_l) - \overline{\alpha})^2 ] \\
	&= C(z_l) + \sum_{i=1}^n a_{li}^2 \left[ (1-\gamma) \kappa_i^2 + \frac{\xi_i^2}{z_{li} } \right] + \gamma \sum_{i=1}^n (a_{li} - a_i)^2 \kappa_i^2
\end{align*}
